C++ 真的和 Java、C\#、PHP、JavaScript 或 Python 一样容易学习吗?尽管语言已经有了诸多改进,答案是:很可能不是。更重要的问题是:C++ 应该像这些语言一样容易学习吗?

C++ 的消亡其实已经被预言了非常长的时间。Java、随后的 C\#,以及如今的 Rust 都曾被宣传为对我们这位备受争议的老将的完整替代品。然而事实是,它们似乎各自开辟了自己的生态领域,而 C++ 仍在那些需要精细优化或运行在资源受限环境中的程序中占据主导地位。更何况,如今仍有数以百万计的 C++ 代码行存在,其中一些已有几十年历史。虽然其中一部分可以转化为云原生、无服务器或微服务架构,但总有一些问题更适合用 C++ 所代表的工程风格来解决。

因此我们得出结论:C++ 在开发世界中有其独特存在的意义,任何一门新的编程语言若想取代它,都将面临一场艰难的挑战。这一观察也带来了相应的结果:C++ 中某些特定的部分,注定会比其他语言更难掌握。

Java 或 C\# 会屏蔽你对内存分配的思考,以及当你把参数传递给另一个方法时内存会发生什么;而 C++ 必须正面应对这些问题,并赋予你根据上下文优化代码的选项。

因此,如果你想真正理解 C++,就无法回避内存管理。幸运的是,它已远不像从前那样是个大问题。

让我们通过比较不同语言如何管理内存分配与释放来分析这些差异。

Java 使用完全的面向对象(OO,object-oriented)方式,其中每一个值都是一个对象。C\# 的设计者则采用了值类型和引用类型并存的方式:值类型包括常见的数值、字符、结构体和枚举,引用类型则对应对象。在 Python 中,每个值也都是对象,其类型可以在程序后续执行中确定。这三种语言都有垃圾回收机制(Garbage Collector)来处理内存的释放。Python 还额外使用了引用计数机制,因此垃圾回收器也可以选择性地被禁用。

而 C++98 标准则没有提供任何内建的指针释放机制,相反,它将内存管理的全部权力和责任交给了开发者。不幸的是,这导致了许多问题。假设你初始化了一个指针,并为其分配了一块较大的内存空间。然后你将这个指针传递给了其他方法,那么谁该负责释放这块内存呢?例如,请看以下简单的代码示例:

\begin{cpp}
BigDataStructure* pData = new pData();
call1(pData);
call2(pData);
call3(pData);
\end{cpp}

调用者应该释放 pData 中分配的内存吗?是 call3 来做这件事吗?如果 call3 又将同一个 pData 实例传递给了另一个函数呢?谁来负责释放内存?如果 call2 执行失败会发生什么?

由此可见,内存释放的责任并不明确,因此必须为每一个函数或更准确地说每一个作用域,明确定义谁负责释放内存。随着程序和数据流复杂度的增加,这个问题的复杂性也会随之上升。大多数使用其他主流语言的开发者面对这种情况要么会感到困惑,要么干脆忽略责任,最终导致内存泄漏,或者访问已经被释放的内存区域。

Java、C\# 和 Python 通过完全不需要开发者小心处理内存的方式解决了所有这些问题。它们依靠两种关键技术:引用计数(reference counting) 和 垃圾回收机制(garbage collection)。引用计数的工作方式如下:每次复制值时,引用计数就递增;当变量离开其作用域时,引用计数递减;当引用计数变为零时,释放内存。垃圾回收器原理类似,但它是定期运行的,并且还会检查循环引用的情况,确保即使复杂的内存结构也能被正确释放,尽管可能会有些延迟。

事实上早在 2000 年代,我们就可以在 C++ 中实现引用计数机制。这种设计模式被称为智能指针(smart pointers),它让我们可以少操心这些底层问题。

其实,C++ 从一开始还提供了另一种更优雅的方式来应对这个问题:按引用传递(pass-by-reference)。这就是为什么 Java、C\# 和 Python 将按引用传递作为默认的对象传递方式:这是一种非常自然且方便的做法。你可以创建一个对象,分配内存,然后按引用传递——最棒的是:当它离开作用域时,内存会自动释放。下面我们来看一个与使用指针类似的示例:


\begin{cpp}
BigDataStructure data{};

call1(data);

call2(data);

call3(data);

...

void call1(BigDataStructure& data){
  ...
}
\end{cpp}

这一次,call1 中发生了什么都已不再重要;只要离开了 data 初始化所在的作用域,内存就会被正确释放。引用类型的唯一限制是:变量所分配的内存无法重新分配。就我个人而言,我认为这其实是一个巨大的优势,因为修改数据时很容易让事情迅速变得复杂;事实上,如果可能的话,我更倾向于使用 const\& 来传递每一个值。当然,也有一些需要通过内存重新分配来实现的、高度优化的多态数据结构,它们的应用场景非常有限。

再来看前面的程序,如果我们忽略掉 call1 中的 \& 符号,并将函数重命名为符合 Java 或 C\# 命名习惯的形式,那么这段代码看起来也可以像 Java 或 C\# 的写法。因此,C++ 从一开始就可以与这些语言非常接近。那为什么现在它仍然不够相似呢?

好吧,在 C++ 中,你无法回避内存管理这一现实。前面这段代码不会让一个 Java 或 C\# 开发者多想什么,但我们已经明确了:C++ 是不同的。标准化委员会也意识到,有些情况下我们需要在一个函数中分配内存,而在另一个函数中释放它,而理想情况下应避免直接使用指针来完成这一任务。于是,移动语义(move semantics) 应运而生。

\begin{myNotic}{Note}
移动语义(Move semantics)是 C++11 引入的一项关键特性,用于通过消除不必要的对象拷贝来提升程序性能。它允许将资源从一个对象转移到另一个对象,而无需创建副本,这对于管理动态内存、文件句柄或其他资源的对象尤其有益。要使用移动语义,你需要为你的类实现一个移动构造函数(move constructor),它通过将资源从一个右值(临时对象)转移到新对象来初始化该对象;以及一个移动赋值运算符(move assignment operator),它将资源从一个右值转移到已有的对象。std::move 是一个工具函数,它可以将一个对象转换为右值引用,从而启用移动语义。值得一提的是,编译器在某些条件下会自动生成移动构造函数。
\end{myNotic}

在下面的示例中,请参阅我们如何使用 move 语义将变量的范围移动到函数进程:

\begin{cpp}
BigDataStructure data{};

process(data);

...
void process(BigDataStructure&& data){
}
\end{cpp}

看起来差别似乎不大,只是多用了两个“\&”符号而已。然而,其行为却大不相同。变量 data 的作用域被移动(move)到了被调用的函数中,接着传入 process,而内存会在离开该作用域时自动释放。

移动语义让我们可以避免复制大型数据,并将内存释放的责任一并转移给被调用的函数。这是目前我们所讨论的语言中,C++ 所独有的一个机制。据我所知,唯一另外实现了类似机制的编程语言,是另外两门面向系统编程的语言:Rust 和 Swift。

这向我们证明了一点:尽管如今的 C++ 看起来越来越像 Java 或 C\#,但它仍然要求开发者更深入地理解内存分配和释放的工作原理。我们或许已经不再需要面对那些专注于语法细节、却影响重大的考试题,但我们仍需学习比其他语言更多的知识。

内存管理虽然是学习过程中的一大难点,但它并不是让 C++ 更难掌握的唯一因素。还有一些特性对于新手来说可能略有不同,甚至有些令人烦恼:

\begin{itemize}
\item 
需要 \#ifndef 预处理器指令或非标准但通常支持的 \#pragma 一次,以确保文件只包含一次

\item 单独的 .h 文件以及 .h 和.cpp 的任意规则
\item 使用virtual methodName()=0 定义接口的非常奇怪的方法
\end{itemize}

尽管我们可以依靠现代 IDE 自动应用各种规则和指南,确保我们正确使用所有这些复杂的机制,但它们的存在本身也引发了一个问题:为什么我们仍然需要它们?

抛开上述问题不谈,还有一个更难接受的事实是:构建程序并添加外部依赖并没有一个简单的方法。Java 虽然有不少缺点,但它拥有统一的编译器,以及 Maven/Gradle 这样标准的依赖管理工具,只需一个简单的命令就可以下载并集成一个新的库。C\# 虽然长期以来也存在类似问题,但现在社区广泛使用的 NuGet 已基本成为获取外部库的标准方式。Python 则提供了标准的 pip 命令用于包管理。

而在 C++ 中,你需要做更多的工作。不像 Java 和 C\# 依赖虚拟机运行,你的 C++ 程序必须为每一个支持的目标平台单独编译,并且每个目标平台都必须匹配相应的库。当然,我们也有一些可用的工具。我最常听到的两个包管理器是 Conan 和 vcpkg,而构建系统方面,CMake 似乎非常流行。问题是:这些工具中没有一个是官方标准。虽然 Java 的 Maven/Gradle 和 C\# 的 NuGet 最初也不是标准工具,但它们被开发工具快速集成和广泛采用,因此如今已经成为事实上的标准。C++ 在这方面的生态系统还有很长的路要走。我们会在单独的章节中进一步讨论这些问题。但很明显,尝试运行一些简单程序时所面临的复杂性,也是造成 C++ 学习困惑的部分原因。

我们已经比较了 C++ 相比其他语言的各种复杂之处,也可以看到,虽然 C++ 已经变得更容易了,但它仍然不像 Java 或 C\# 那样简单。但核心问题依然是:C++ 真的很难学吗?为了探讨这个问题,让我们来看看初学者学习 C++ 可以采用的三种方法。


