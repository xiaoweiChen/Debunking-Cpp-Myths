在本章中,我们了解到C++已从其源自C(以及B和BCPL)的朴素起源中显著进化并分化。我们探讨了C++如何通过引入现代特性和更严格的规则来提升安全性与效率,并支持现代编程范式。尽管保留了C的大部分语法,但两种语言随时间推移已严重分叉,导致兼容性挑战——尤其是在将传统C代码与需要新C++标准的功能混合使用时。本章对此进行了深入讨论。

在现代C++自身演进中,移动语义的引入、模板解析的严格化以及auto等关键字行为的改变,都为语言增添了新的复杂度(虽然本来的复杂度就已经够高了)。这也是我们在本章学到的重点。

尽管存在这些挑战,但我们发现C++仍在延续其丰富遗产的基础上不断发展:既为开发者提供强大工具,又要求他们谨慎对待标准演进与向后兼容性的平衡,同时避免与早期版本产生过多矛盾。它始终是一门让传统与创新交汇的语言——且往往以出人意料又引人入胜的方式交汇。

但这种平衡能维持多久?面对后起之秀的威胁,C++能否继续生存?Rust会取代C++吗?亲爱的读者,这取决于您的选择,而Alex将在下一章为您展开深度剖析。