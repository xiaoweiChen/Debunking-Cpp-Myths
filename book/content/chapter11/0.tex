\begin{flushright}
\textit{--- 当然,说到C语言,就不得不提B语言……甚至还有A语言……以及@符号?}
\end{flushright}

最初,编程语言的圣言记载于BCPL(Basic Combined Programming Language的缩写,可不是巴尔的摩县公共图书馆)。作为编译器领域的初代霸主,它以铁血语法统治了数个迭代版本。然而岁月无情,新特性、新范式、新语法层出不穷,很快,一位新的王位继承者从比特海中崛起:B语言。但世人鲜少理解B语言"无类型"设计的精妙之处,随着新的编程语言王者候选者——C语言\footnote{\url{https://www.bell-labs.com/usr/dmr/www/chist.html}}的出现,B语言很快退出历史舞台。

后来的故事众所周知。C语言成为系统级编程的事实标准,其语法结构渗透到过去及本世纪所有主流编程语言中(比如随处可见的花括号)。它如同计算机王国中的圣胶,将各类编程语言粘合在一起执行神圣仪式。开发者们见证这一切,并称其为美好。

直到一位普罗米修斯式的人物\footnote{没错,斯特劳斯特鲁普(Bjarne),我们说的就是你呢。}出现——他将"类"的概念引入C语言,先是创造了"带类的C"(C with classes),随后开发出史上首个C++编译器Cfront(它能将C++代码转换为C代码,可惜已湮灭在时光长河中)。但它的遗产永存:这门语言本身、数十个在不同时期(或多或少)符合标准的C++编译器、数百种未定义行为案例,以及过去三十年间迭代的各个标准版本(最新可用的是C++23,而标准委员会正在酝酿更强大的C++26)——所有这些共同构成了我们挚爱的编程语言:C++。

本章将用以下话题让你如遇早高峰般寸步难离(却又欲罢不能):

\begin{itemize}
\item 
C++真的向后兼容C吗?

\item 
C++真的向后兼容C++自身吗?
\end{itemize}

























