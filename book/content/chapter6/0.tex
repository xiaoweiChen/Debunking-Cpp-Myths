\begin{flushright}
\textit{--- 若您仍以2000年代的风格编写C++代码}
\end{flushright}


C++在安全性方面确实存在一些问题,内存安全便是其中之一。内存安全问题主要分为两类:空间安全问题(比如:访问越界内存)和时间安全问题(比如:访问状态不确定或已释放的内存);现代C++通过以下方式规避这些风险:避免使用裸指针(naked pointers)和std::span,以及使用概念(concepts)。然而仍需改进:本章将展示当前C++机制仍不完善,并探讨安全规范(safety profiles)作为未来可能的改进方向。

本章将涵盖以下主题:

\begin{itemize}
\item 
内存安全的重要性

\item 
传统C++的内存安全问题

\item 
现代C++的解决方案

\item 
现代C++的局限性

\item 
仍需改进之处
\end{itemize}














