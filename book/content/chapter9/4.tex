
是时候坦白了 --- 笔者已厌倦了用丑陋代码角逐 C++ 选美大赛的行径。无论我们如何自我催眠,坚称前文展示的那些代码多么精妙绝伦、值得铭记,事实终究无法掩盖:它们丑陋不堪。

亲爱的读者,请您忘记自己曾被迫读过这些“代码噩梦” --- 在此,我们深表歉意。

从这一刻起,我们郑重承诺:

不再胡闹,只为您呈现优雅、纯净、可爱的 C++ 代码。

作为这场代码审美革命的起点,我们现在自豪地宣布:我们已经正式加入“优雅 C++ 代码书写者”的行列!

接下来要展示的程序,可能是你所能邂逅的最可爱 C++ 示例:

\myGraphic{0.8}{content/chapter9/images/0.png}{}

为求简洁,我们省略了对 std::string、std::cout 和 std::unique\_ptr 的头文件引用 --- 谁说 C++ 不能可可爱爱?

可惜的是,这段代码尚未被广泛接受为标准 C++(似乎编译器开发者们仍在争论哪些 Unicode 标识符是合法的,尽管最新 C++ 标准中确实存在 [tab:lex.name.allowed] 这样的条款)。

但希望并未泯灭 --- 毕竟,GCC 接受了它!或许它的开发团队里藏着一位熊类爱好者。

顺带一提,这段代码功能简单:根据营养需求、膳食限制及熊之国的饮食风尚,为各类熊提供定制餐食。这难道不是个既可爱又有资格角逐"最美C++代码"的范例吗?

若您希望编写符合常识、可读性强、稳定易维护且遵循最新标准的程序,我们推荐诸多优秀著作。可惜这些书鲜少提及如何编写趣味程序 --- 因为趣味编程需要截然不同的心态,往往与商业收益无关。编程可以成为艺术,用彩蛋、幽默输出、趣味交互或非常规可视化给人惊喜。就像我们用emoji作标识符,或是构建古怪逻辑的应用程序。

趣味编程常打破常规,为创意解法牺牲效率,比如故意编写晦涩代码只为乐趣。它也可以是解谜题、探索非常规编程范式(如函数式或Brainfuck等冷门语言),或纯粹出于好奇心的项目。

当传统"优美代码"强调正确性、安全性和可读性时,趣味编程则为 spontaneity、艺术性和娱乐性敞开大门。正因如此,本章最后一个示例选择了童话般的熊主题短代码---毕竟,谁能拒绝可爱的熊呢?