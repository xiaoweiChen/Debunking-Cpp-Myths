
是时候坦白了——笔者已厌倦了用丑陋代码角逐C++选美大赛的行径。无论我们如何自我催眠,坚称前文展示的那些代码多么精妙绝伦、值得铭记,事实是:它们丑陋不堪。亲爱的读者,请忘记您曾被迫阅读过那些代码,在此深表歉意。

从此刻起,我们郑重承诺:不再胡闹,只为您呈现优雅代码。不再有丑陋的宏定义,不再有晦涩的替换技巧,不再有神秘的黑魔法——唯有纯粹、愉悦、可爱的C++。

作为这场代码美学的自我革新(现在我们是优雅C++代码的书写者了),接下来展示的程序可能是您所能邂逅的最可爱示例:

\myGraphic{0.9}{content/chapter9/images/0.png}{}

为求简洁,请容我们省略了对std::string、std::cout和std::unique\_ptr的头文件引用——谁说C++不能可可爱爱?

可惜这段代码并未被广泛认可为标准C++(似乎编译器开发者们对哪些Unicode标识符合法尚未达成共识,尽管最新C++标准中有[tab:lex.name.allowed]条款)。但希望未泯,毕竟GCC接受了它,或许其开发团队里藏着位熊类爱好者。

顺带一提,这段代码功能简单:根据营养需求、膳食限制及熊之国的饮食风尚,为各类熊提供定制餐食。这难道不是个既可爱又有资格角逐"最美C++代码"的范例吗?

若您希望编写符合常识、可读性强、稳定易维护且遵循最新标准的程序,我们推荐诸多优秀著作。可惜这些书鲜少提及如何编写趣味程序——因为趣味编程需要截然不同的心态,往往与商业收益无关。编程可以成为艺术,用彩蛋、幽默输出、趣味交互或非常规可视化给人惊喜。就像我们用emoji作标识符,或是构建古怪逻辑的应用程序。

趣味编程常打破常规,为创意解法牺牲效率,比如故意编写晦涩代码只为乐趣。它也可以是解谜题、探索非常规编程范式(如函数式或Brainfuck等冷门语言),或纯粹出于好奇心的项目。

当传统"优美代码"强调正确性、安全性和可读性时,趣味编程则为 spontaneity、艺术性和娱乐性敞开大门。正因如此,本章最后一个示例选择了童话般的熊主题短代码——毕竟,谁能拒绝可爱的熊呢?