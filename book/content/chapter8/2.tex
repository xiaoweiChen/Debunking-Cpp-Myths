
最尊敬的读者。众所周知,所有开发者在职业生涯中都必须经历技术面试这一关。面试的难度各有不同:有些仅停留在"请做个自我介绍"的层面(这些往往最难应对),有些则更为深入,可能要求你在黑板甚至电脑上编写代码。

在面试题中频繁出现的一个经典程序是:编写计算具有特定特征的数列之和的代码,例如所有偶数的和、能被五整除的数字之和,或是特定区间内奇数的总和。

为简化起见,我们以一道简单题目为例:计算100以内所有奇数的和。以下这段简洁的程序完美实现了这个功能:

\begin{cpp}
#include <cstdio>
int main() {
  int sum = 0;
  for (int i = 1; i <= 100; ++i) {
    if (i % 2 != 0) { // Check if the number is odd
      sum += i; // Add the odd number to the sum
    }
  }

  printf("The sum is: %d\n",sum);
  return 0;
}
\end{cpp}

这个程序并不复杂:只需遍历数字,检查是否为奇数;若是,则将其值累加到总和中;最后输出结果(感兴趣的读者请注意,1到100之间所有奇数的和正好是2,500)。

然而,我们的清晰思路被一个众所周知的事实(至少对C++程序员而言)所干扰:最快的C++代码往往内嵌汇编。于是,我们决定在速度的祭坛上牺牲程序的可移植性和可读性,用汇编语言重写其核心部分。毕竟,这才是最快的实现方式。以下是我们采用AT\&T汇编语法进行的尝试,仅用于展示我们可以在不符合标准规范的C++程序中嵌入的各种常见汇编方言:

\begin{cpp}
#include <cstdio>

int main() {
  int sum = 0;
  int i = 1; // Start with the first odd number
         
  __asm__ (
    "movl $1, %[i]\n" // Initialize i to 1
    "movl $0, %[sum]\n" // Initialize sum to 0
    "loop_start:\n"
    "cmpl $100, %[i]\n" // Compare i with 100
    "jg loop_end\n" // If i > 100, exit the
    "addl %[i], %[sum]\n" // sum += i
    "addl $2, %[i]\n" // i += 2
    "jmp loop_start\n" // Repeat the loop
    "loop_end:\n"
    : [sum] "+r" (sum), [i] "+r" (i)
  );

  printf("The sum is: %d\n", sum);
  return 0;
}
\end{cpp}

以下简要说明汇编代码的功能,因为其他代码行应该能自解释:

汇编代码解析:

\begin{enumerate}
\item 
\verb|"movl $1, %[i]\n"|:将 i 设为 1。尽管 i 已在 C++ 代码中初始化为 1,但这里显式地在汇编中再次设置,以确保清晰。

\item 
\verb|"movl $0, %[sum]\n"|:将 sum 设为 0,确保在汇编代码中总和从 0 开始计算。必须承认,这两步初始化并非必要,但我们希望它们能温和地引导你进入汇编世界,避免一开始就吓跑你。

\item 
\verb|loop_start:|:仅是一个标签,无需额外解释。

\item 
\verb|"cmpl $100, %[i]\n"|:比较 i 和 100,用于检查 i 是否达到或超过 100。

\item 
\verb|"jg loop_end\n"|:若 i 大于 100,则跳转至 loop\_end,退出循环。

\item 
\verb|"addl %[i], %[sum]\n"|:将当前 i 的值累加到 sum 中,逐步计算 1 至 99 的奇数和。

\item 
\verb|"addl $2, %[i]\n"|:将 i 增加 2,跳转到下一个奇数(如 1 -> 3 -> 5,依此类推)。

\item 
\verb|"jmp loop_start\n"|:跳回循环起始处,重复执行。

\item 
\verb|loop_end:|:当 i 超过 100 时,程序跳转至此标签,循环结束。
\end{enumerate}

关于 \verb|"+r" (sum)| 和 \verb|"+r" (i)|:这些是约束标记,告知编译器将 sum 和 i 视为可读写变量,即在汇编操作中既可读取也可修改其值。

第一个缺点:代码可读性与可理解性急剧下降。我们故意使用 AT\&T 汇编语法,因为它更繁琐、更难懂,目的是让你亲身体会其痛苦,并铭记——除非你清楚自己在做什么(即便如此也需谨慎),否则永远别在代码中嵌入汇编。

第二个缺点:代码不再具备可移植性。
因为 \verb|__asm__| 在 Visual C++ 中并不存在(早年使用 \verb|__asm|,或更近期的 Turbo C 演示了 asm 关键字的引入)。此外,C++ 标准并未统一内联汇编的语法,因为汇编语言依赖于编译器和平台,内联汇编属于扩展功能而非语言核心。你已被警告。

希望上述说明能彻底打消你在 C++ 函数中直接编写汇编代码的念头——无论是否存在非标准关键字允许你这样做。

但既然我们已经借助 gcc.godbolt.org 来到了这一步,我们便让各大主流编译器在不同优化级别下处理最初那个简单的 C++ 程序(完全不涉及汇编),因为我们迫切想向你证明——在此阶段完全跳过汇编语言,才是最明智的选择。

第一个展示编译器高效生成优化代码能力的是 Microsoft Visual C++。微软自家小巧灵活的 C++ 编译器提供了多种代码生成和优化选项\footnote{\url{https://learn.microsoft.com/en-us/cpp/build/reference/o-options-optimize-code?view=msvc-170}},但我们信奉一条准则:代码越短,运行越快。因此,我们明确要求编译器生成最短的代码(/O1),结果如下:

\begin{shell}
`string' DB 'The sum is: %d', 0aH, 00H ; `string'
_main PROC
  xor ecx, ecx  ; Clear the ECX register (set ECX to 0)
  xor edx, edx  ; Clear the EDX register (set EDX to 0)
  inc ecx       ; Increment ECX, setting it to 1
  $LL4@main:
  test cl, 1    ; Test the least significant bit of CL
                ; (ECX) to check if ECX is odd or even
  lea eax, DWORD PTR [ecx+edx] ; Load the effective
                ; address of ECX + EDX into EAX
  cmove eax, edx; If the zero flag is set
                ; (ECX was even), move EDX into EAX
  inc ecx       ; Increment ECX by 1
  mov edx, eax  ; Move the value in EAX to EDX
                ; (update EDX for the next iteration)
  cmp ecx, 100  ; Compare ECX with 100
  jle SHORT $LL4@main ; Jump to the start of the loop
                ; (loop until ECX > 100)
  push edx      ; Push the final value of EDX (the sum)
                ; after the loop onto the stack
  push OFFSET `string' ; Push the offset of the string
  call _printf  ; Call the printf function
  pop ecx       ; Clean up the stack (remove string)
  pop ecx       ; Clean up the stack (remove EDX)
  ret 0         ; Return from the _main function
_main ENDP
\end{shell}

趣的是,MSVC 生成的汇编输出与我们手工编写的版本高度吻合——它同样采用循环结构,只是根据当前处理数字的奇偶性对寄存器的操作略有差异,但除此之外,其逻辑与我们编写的代码几乎一致。

有趣的是,即使尝试了 MSVC 的其他优化选项组合(/Ox、/O2 和 /Ot),生成的代码也并无显著差异——仅仅是寄存器分配略有不同,完全达不到让人惊叹"哇!"的程度。而当我们切换到 GCC (14.1) 来编译这段简单代码时,发现 -O1 和 -O2 优化级别下生成的代码与 MSVC 极为相似:通过变量遍历数字,进行奇偶判断并累加。

毫无黑魔法痕迹...直到我们启用 -O3 优化。这个标志位让编译器祭出了单指令多数据流(SIMD)指令集来加速运算:它竟然将初始值为 \{1, 2, 3, 4\} 的4元素数组通过25次迭代(每次元素值递增4)进行SIMD并行求和,最终将SIMD寄存器中的累加结果规约为单个整数输出。这段长达三页多的汇编代码因实用性过低未被收录,但作为趣味知识值得提及。

接下来测试的 Clang 编译器给了我们惊喜。在见识过GCC -O3冗长的SIMD指令后,我们本不抱期待,但即便在 -O1 级别,Clang 生成的代码简洁得令人意外:

\begin{shell}
main:
  push rax
  lea rdi, [rip + .L.str]
  mov esi, 2500
  xor eax, eax
  call printf@PLT
  xor eax, eax
  pop rcx
  ret
.L.str:
  .asciz "The sum is: %d\n"
\end{shell}

令人惊叹!Clang 似乎直接在编译阶段就完成了所有计算,仅仅将结果硬编码到二进制文件中——这已然是优化的极致。我们由衷感慨编译器已进化成如此精妙的"猛兽",这促使我们好奇其他编译器是否也能这般聪明。

GCC 在 -O3 优化级别下展现了相同行为,但令人意外的是:仅当计算 71 以内的奇数和时会直接输出结果。一旦上限改为 72,其内部机制就会崩溃,重新生成冗长的 SIMD 汇编代码。

而 MSVC 则始终无动于衷——无论我们如何尝试不同的数值和参数组合,它都拒绝像 Clang 那样预计算奇数和。我们只能遗憾判定:当前版本确实不具备此能力。微软 Visual C++ 的开发者们,或许下一个版本能实现这个功能?您觉得呢?

\mySubsubsection{8.2.1}{预见未来}

在C++开发者圈子里流传着这样一句话:如今的编译器优化是我们迄今为止拼凑出的最佳成果——同时也是对它们还能变得多强大的无情提醒。

考虑到本书写于2024年(如果顺利的话将在2025年出版,而按照计划到2027年就会过时,届时我们将受命编写更新版本),我们对当下技术发展有着清晰的认知。

不过,当你读到这本书时,如果人类已经在其他星球种植土豆,而你所在建筑的墙面上爬满了涂鸦的机械猴——那时你或许已经见证了编译器在过去十年间的长足进步。说不定微软那个(没错,就是我们说的那个小巧软萌的)C++编译器已经成长到能在编译前计算简单数列和,而GCC面对数字72也不会突然"发脾气"了。即便对于我们演示的这种简短程序也是如此。

欢迎来到未来。








