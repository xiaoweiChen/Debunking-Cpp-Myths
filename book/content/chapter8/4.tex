C++编程神话的形成源于这门语言不断演进的历史轨迹、开发者群体间参差不齐的掌握程度,以及程序员社区特有的心理诉求。早期C++编译器生成的代码效率往往逊色于现代编译器,由此催生了关于"必须手动优化"的种种传说——比如用平台特定的汇编语言重写整个函数模块这类如今看来过时的实践。

尽管编译器和语言特性已突飞猛进,这些神话依然顽固存在,甚至时常掩盖现代最佳实践。加之C++社区固有的精英主义文化和技术优越感,使得某些过时认知被不断强化。有趣的是,即便在这样的语境下,C++始终保持着高性能关键应用领域首选语言的王者地位。

下一章我们将举办编程语言选美大赛,经过层层淘汰后,最终加冕的无冕之王自然是——(此处鼓点渐强)C++。坦白说,我们对这门语言的偏爱如此明显,读者或许早该察觉这场选美从开始就内定了冠军。