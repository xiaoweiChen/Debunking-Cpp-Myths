C++ 编程神话的形成,源于这门语言漫长而不断演进的历史轨迹、开发者群体之间参差不齐的技术掌握程度,以及社区中普遍存在的心理诉求 --- 对性能极致掌控的执念与技术优越感。

在早期,C++ 编译器生成的代码效率往往逊色于手工优化的版本,由此催生了大量关于“必须手动优化”的传说:比如为了追求极致性能,开发者不惜用平台相关的汇编语言重写整个函数模块。这些如今看来已显过时的做法,在当时却成为塑造 C++ 神话的重要素材。

尽管现代编译器和语言特性早已突飞猛进,许多曾经的“真理”却依然顽固地流传至今,甚至掩盖了更高效、更安全的现代最佳实践。再加上 C++ 社区特有的精英主义文化,使得某些落后的认知被反复强化,仿佛不写点汇编就不好意思说自己是“真正的系统程序员”。

有趣的是,在这样的背景下,C++ 依旧稳居高性能关键应用领域的王者地位。它不仅存活了下来,还在嵌入式系统、游戏引擎、高频交易、操作系统开发等领域持续占据核心位置。

下一章我们将举办编程语言选美大赛,经过层层淘汰后,最终加冕的无冕之王自然是 --- (此处鼓点渐强)C++。

坦白讲,我们对这门语言的偏爱早已藏不住了。或许读者早在本书前几章就该察觉:这场选美比赛从一开始,冠军就已经内定。