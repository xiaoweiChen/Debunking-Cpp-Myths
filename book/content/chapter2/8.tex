正如本章所展示的那样,编写符合标准的 C++ 代码可以确保代码在不同平台和编译器之间的可移植性、兼容性和可维护性。我们了解到,遵循 ISO/IEC C++ 标准,我们可以编写出行为可预测、更少出现 bug 和平台相关问题的代码。符合标准的 C++ 代码还能从通用的编译器优化中受益,并适应未来的语言增强功能,从而保证代码的长期可用性和性能,这也是我们在本章中学到的内容。

另一方面,使用特定于 C++ 编译器的扩展可以带来针对特定平台和编译器的性能优化、对尚未标准化的高级特性的访问能力,以及与厂商特定工具的集成。然而,这些扩展可能会引入可移植性问题、依赖特定版本的编译器,并导致与标准 C++ 实践的偏离,从而影响代码在不同平台和编译器间的维护和互操作性,这也是本章所涵盖的内容。

因此,我们了解到,是否采用这些扩展应根据项目的实际需求慎重考虑,权衡增强功能带来的好处与潜在的兼容性和长期支持问题。在这个阶段,我们相信你现在有能力做出最适合你的项目和代码库的决策,让你顺利交付所需的产品——即使这是一个你在一台 30 年前的老机器上闲暇时写的个人项目,用一个 30 年前的编译器编译的程序。

接下来的章节由 Alex 呈献,我们将深入探索并试图揭开一个基本真相:C++ 究竟只是另一种面向对象的语言,还是在其表面之下隐藏着更深层的奥秘……