正如本章所展示的那样,编写符合标准的 C++ 代码是确保其在不同平台和编译器之间具备可移植性、兼容性与可维护性的关键所在。

我们已经了解到,遵循 ISO/IEC C++ 标准,不仅可以帮助写出行为可预测、更少出现 bug 和平台相关问题的代码,还能使程序从通用的编译器优化中受益,并更好地适应未来的语言演进。这意味着代码不仅能在今天运行良好,在多年之后依然能够保持良好的可读性与性能表现 --- 这正是我们在本章中反复强调的核心价值。

然而,另一方面,使用特定于编译器的扩展也带来了不容忽视的优势:它们为开发者提供了针对特定平台的性能优化能力、对尚未标准化的高级特性的早期访问权限,以及与厂商专属工具链的深度集成。这些功能往往能显著提升开发效率或程序性能。

但与此同时,这些非标准扩展也可能引入一系列潜在问题:包括但不限于降低代码的可移植性、增加对特定编译器版本的依赖、以及偏离主流 C++ 实践所带来的长期维护风险。这也是本章所重点探讨的内容。

因此,我们得出结论:是否采用这些扩展,必须根据项目的实际需求进行慎重权衡。需要在功能增强带来的短期收益与兼容性、可维护性和长期支持方面的潜在成本之间做出取舍。

在这个阶段,我们相信你现在已具备足够的判断力,能够为项目和代码库做出最合适的决策 --- 无论是为了交付一个商业产品,还是出于兴趣,在一台三十年前的老机器上,用一个三十年前的编译器,都能完成个人的小项目。

接下来的章节将由 Alex 呈现。我们将深入探索并试图揭示一个根本性的问题:C++ 究竟是仅仅另一种面向对象的语言?还是在其表面之下,隐藏着远比我们所见更深邃、更丰富的编程范式与设计哲学?