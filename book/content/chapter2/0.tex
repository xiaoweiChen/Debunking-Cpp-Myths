
\begin{flushright}
\textit{--- 除非它们并非如此}
\end{flushright}

在 C++ 编程领域,标准合规性的概念通常受到高度重视,C++ 标准的最新版本被视为编写正确和高效代码的权威指南。C++ 标准由 C++ 委员会和国际标准化组织 (ISO) 精心制定并定期更新,是开发人员的终极指南,提供了一套全面的规则和最佳实践,以确保代码质量和互作性。 然而,软件开发的现实比这个理想所暗示的要微妙和复杂。

在本章中,我们将深入探讨开发人员面临的无数挑战,这些挑战由于各种限制,无法始终遵守这些标准,并仔细平衡理想标准和实际工作需求之间的微妙平衡。这些约束可能包括其开发环境中的限制,例如过时的编译器、遗留系统或强制使用非标准功能的特定项目要求。

当我们被迫使用使用 C++ 作为基础并提供一组扩展来满足特定用例的框架时,可能会出现复杂的情况。正如我们将在后面介绍的那样,这些框架构建在现有的标准 C++ 之上,并引入了针对特定范围的高度特定的功能,但与 C++ 标准没有任何共同之处。所以,我们可能
会问自己:我们应该使用这些框架吗?正如我们将看到的,这个问题的答案并不像人们想象的那么简单。

在本章中,我们将涵盖以下主要主题:

\begin{itemize}
\item 
各种编译器、框架和环境中对标准的遵循

\item 
为什么不是所有人都能学习、使用或编写符合标准的 C++?

\item 
编译器扩展逐渐偏离标准的现象
\end{itemize}