当 Richard Appiah Akoto 在加纳学校的黑板上发布几张自己在 Microsoft Word 用户界面上绘的照片时,他立即在一夜之间成为社交媒体现象1。他的学校很穷,没有能用的电脑,只一块世纪之交的标准黑板,但这并没有阻止他履行教师的职责。他以非常有创意的方式尽最大努力向他的学生传达改变生活的知识,希望有一天,这些知识对他们追求更好的活有用。剩下的就是历史了,但真正的问题是:这是教授 Microsoft Word\footnote{\url{https://news.microsoft.com/apac/2018/03/17/teacher-who-used-a-chalkboard-in-computer-class-because-he-had-no-computer-stars-at-microsofts-education-exchange/}} 的标准方式吗?

我们不要与我们最初的目标相差太大。我们想了解 C++ 程序的标准合规性。对于狂热的 C++ 开发者来说,该标准的最新版本被视为神圣的经文、单词、他们应该遵守的规则集合, 何偏离它的行为都应该被删除和重写不符合标准的代码。或者在拘留中心被贴上遗留代维护者的标签一周。

面对严酷的现实,事情离理想主义的环境不远了。一些开发人员无法使用最新版本的 C++ 准。这可能是因为他们的生计与需要特定编译器的现实生活中的项目绑定在一起,或者为他们编程的环境不允许使用语言的特定功能。

或者,也许他们通过在一个过去 20 年根本没有收到更新的平台上工作而获得报酬,因为该供商在十年前宣布破产,而且没有人接手他们的业务。但是,由于一切都正常并且仍然生收入,因此它是使用 20 年前可用的工具进行保存和维护的。

这绝对不包括支持最新 C++ 标准的编译器。那么,这是否意味着在这些平台上工作的开发员编写的 C++ 代码不符合标准? 在世纪之交,本章的作者发现自己在大学的一间教室里,参加了一门名为 C++ 编程简介的程。这是那个地方唯一开设的 C++ 课程,老师用一本书将知识传授给 30 多个学生。

路尽头的小复印店老板非常高兴,因为有一天,老师决定把这本书借给其中一个学生。这书是 Kris Jamsa 的 C/C++ 开发者圣经的翻译和大量简化版本,我们称之为“有斑点狗的书”。

该书的本地版本仅包含 C++ 部分,但它带有一个非常重要的插图:Turbo C++ Lite IDE 和随的编译器,位于标准的 1.44 MB 软盘上。对于那些不熟悉这个名字的人来说,Turbo C++ ite 是 Borland 的流行(且非常用户友好)IDE 和编译器 Turbo C++ 的简化版本。编译器是同的,但是,为了在单个 1.44 MB(即兆字节)软盘上适应整个环境,删除了许多功能和工具。

这是我们第一次介绍编译器、链接器和语法的复杂世界。我们中的一些人发现它如此迷人以至于即使在 20 年后的今天,我们仍然在日常工作中使用它。因此,正如您可以想象的样,我们的第一个 C++ 程序类似于下面的屏幕截图中的程序。

\myGraphic{0.9}{content/chapter2/images/1.png}{图 2.1 – 臭名昭著的代码屏幕,如 Life of a Programmer (1997) 中所见}

哦,你脸上的恐惧!我可以清楚地想象,亲爱的c++爱好者。纯粹的看到:

\begin{itemize}
\item 
iostream.h: 嗯,你好,现在是1999年,C++98标准去年就发布了。你这个异端,为什么不使用它?它的全称是ISO/IEC 14882:1998,花区区200瑞士法郎就能买一本。……哦,这对你来说可是兼职当三个月洗碗工才能赚到的学费啊?

\item 
void main(void): 哦天哪,这根本就不是任何C或C++标准里的东西。你刚挖出来的到底是什么黑暗魔法?……难道这就是他们所说的新玩意儿……Java?

\item 
cout: 从未使用过 using 指令,这怎么可能呢?
\end{itemize}

这里,你完全可以放弃试图理解其中的原因,如释重负地叹口气,但请耐心听我说下去。

与理查德·阿皮亚·阿科托 (Richard Appiah Akoto) 面临的条件非常相似,在我们的教育阶,我们也可以使用一间带黑板的教室,如前所述,由一位敬业的老师和这本书以及几本陪伴。即便如此,我们还是学会了 C++。也许,从标准的角度来看,这些都是理想的情,因为 C++ 标准非常宽松,考虑到环境,它不需要现代计算机中能找到的东西——没有盘,没有屏幕,甚至不需要作系统。事实上,唯一非常严格的环境要求是 char 的大小必至少为 8 位。这是为了确保 char 可以保存基本执行字符集(包括标准 ASCII 字符)的任成员。事实上,sizeof(char) == 1 也由 C++ 标准以及它的有符号和无符号版本保证。其一切都建立在这些基础上。

所以,我们可以说,在我们被允许进入计算机实验室之前,我们有学习标准 C++ 的理想环。没有烦人的系统依赖关系,没有计算机崩溃,也没有硬件,以防代码无法编译时遇到败感。由于我们没有在黑板上运行编译器,我们的老师很快意识到在黑板上编译更复杂 C++ 代码并不完全可行,因此我们被分配到了周五清晨在计算机实验室的时间段。所有麻烦都是在那之后开始的。

解释其实很简单:你看,我们学校当年用来教授 C++ 的计算机实验室里,只有一堆 80286 IBM AT 兼容机。

你没看错。30 名学生被分配了八台计算机(每台计算机都配备了出色的 80286 处理器,在们构思时可能是高科技的,尽管在一年半之后已经过时了),这些计算机可能是某个可进行了升级的援助组织的传下来的,并决定将他们的旧设备捐赠给大学,以获得公司学的税收优惠。 四个半人坐在一台机器前,拿着一本书(和几本)供课堂使用,试图学习C++。


尽管情况不像二十年后理查德·阿皮亚·阿科托 (Richard Appiah Akoto) 的学校那样凄凉, 根本没有比这更好的条件了。这些机器只能运行纯 DOS 以外的任何东西,而且没有比 10 前问世的 Turbo C++ Lite 更好的编译器了。这是否意味着我们有意学习编写非标准的 C + 代码?不,显然不是。 我们编写了我们有可能编写的代码。

但是,我们不要回到过去那么久。截至 2024 年,即本书撰写之日,Stack Overflow (https://s ackoverflow.com/) 上有 46 个问题,其中包含可怕的 void main(void) 短语。最新的一个人惊讶地来自 2023 年。iostream.h 内容更多一些,但主要是教育上下文,我们不敢在不遇 using 指令或命名空间限定符的情况下计算包含 cout 的内容,因为这将是徒劳的。 这是否味着即使在 2024 年,仍有开发者编写依赖于非标准 C++ 的代码?或者是否有学生以非标的方式学习 C++? 

进一步深入研究 Stack Overflow,弹出了 C++ 的另一个有趣的旧方言:conio.h。这个头文件在语言正式标准化前几年随 Turbo C(和 C++)一起发布的,但考虑到 2024 年仍有年轻学徒对此提出问题,我们可以说前一个问题的答案极有可能是肯定的。

根据他们的情况和可能性,无论他们是必须使用黑板学习,用粉笔画画,还是通过共享键,在这个过程中轻轻地敲击彼此的手,今天仍然有一些开发者不由自主地被强加于他们习和编写非标准 C++ 的过程。