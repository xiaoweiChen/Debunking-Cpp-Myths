必须坦诚地指出,阅读本章可能不会是一个轻松的过程,但我们将会尽力让内容尽可能易于理解。我们的讨论将穿梭于 C++ 所支持的各种平台、编译器以及不同的“方言”之间。然而,在某个时刻,必须划清理论与实践之间的界限,并尝试将这些看似抽象的概念转化为真正可用的代码。

因此,强烈建议您能够访问互联网,以便随时查阅相关信息。而如果想动手尝试不同的编译器行为,我们特别推荐一个实验性 C++ 探索的首选工具 --- Matt Godbolt 的 Compiler Explorer:

\url{https://gcc.godbolt.org/}

这个在线平台几乎支持我们在本章中将要讨论的所有主流 C++ 编译器,是观察不同编译器如何处理代码、探索语言特性和优化行为的理想场所。

目前,无需下载或保存任何资料,因为我们还没有写出足够有价值或具有代表性的代码,值得提交到本书的 GitHub 仓库中。换句话说,现在所生成的代码只是为了辅助理解,并不需要保留。