在本章中,我们已经看到,C++ 拥有覆盖各种需求的丰富库与框架。然而,与其他技术生态相比,获取这些资源的过程并不轻松:它们缺乏统一集中的存放平台,导致发现困难;更可能引发诸如编译器兼容性不佳、代码风格陈旧等附加问题 --- 这正是本章所揭示的关键认知之一。

正如其他技术栈一样,C++ 的库生态同样面临漏洞风险,并易受供应链攻击的威胁。常见的防范措施包括持续跟踪漏洞披露动态、在下载时验证二进制文件的真实性等。正如此章所述,额外的代码审计与漏洞扫描始终是有益无害的。因此,大型组织在此方面更具优势 --- 他们通常配备有专门的安全团队来应对这些问题,代价却是牺牲了一定的灵活性与敏捷性。

那么,C++ 是否拥有支持现代编程范式的库?答案无疑是肯定的。只是相较于其他主流语言生态,这些库往往更难发现、更难集成、兼容性更差。

在下一章中,我们将进一步探讨一个更为根本的问题:

C++ 是否真正具备良好的向后兼容性?

这不仅涉及语言本身的版本迭代,也关乎其更广泛的生态系统是否能在演进过程中保持稳定与连贯。