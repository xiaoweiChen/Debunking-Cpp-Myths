
我们所处的世界本质上是以并行方式运作的。每一棵树、每一株植物,乃至每一个人都在独立地运行和变化,偶尔通过相互作用来影响彼此的状态。因此,我们天生就具备理解并行程序运作的心智模型:在一个具备同步机制的基础设施之上,各自独立封装的行为实体通过某种方式进行交互。

这一理念启发了 Carl Hewitt 于 1973 年提出 Actor 模型(Actor Model)。该模型将程序拆分为多个能够执行以下三种基本行为的“actor”

\begin{itemize}
\item 
向其他actor发送消息

\item 
创建新actor

\item 
定义处理下一条消息时应采取的行为准则
\end{itemize}

每个 actor 都拥有一个类似电子邮件地址的唯一标识符,它只能与自己已知地址的 actor 进行通信。这些地址通常通过接收消息或创建新 actor 的过程获得。

Actor 模型的一个核心优势在于 --- 它将通信机制与业务逻辑进行了清晰的解耦。这种设计使得开发者无需直接操作线程原语,就能编写出高度并发且可扩展的程序。

在 C++ 社区中,最成熟和稳定的 Actor 模型实现是 CAF(C++ Actor Framework)(\url{https://www.actor-framework.org/})。近年来也出现了新的实现方案,例如阿里巴巴开源的 Hiactor:(\url{https://github.com/alibaba/hiactor})。不过,最广为人知的 Actor 实现来自 Java 生态系统,那就是 Akka 工具包:(\url{https://akka.io/})。

为了展示 Actor 模型的实际应用,来看一个使用 CAF 实现两个 actor 之间简单聊天的示例。在这个例子中,我们将 actor 的行为定义为 Lambda 表达式,实例化两个用于聊天的 actor,并在它们之间发送消息。每个 actor 在接收到消息后会将其内容输出到控制台:

\begin{cpp}
behavior chatter(event_based_actor* self, const string& name) {
  return {
    [=] (const string& msg) {
      cout << name << " received: " << msg << endl;
    }
  };
}

void caf_main(actor_system& system) {
  auto alice = system.spawn(chatter, "Alice");
  auto bob = system.spawn(chatter, "Bob");
  scoped_actor self{system};
  self->send(alice, "Hello Alice!");
  self->send(bob, "Hello Bob!");
  self->send(alice, "How are you?");
  self->send(bob, "I'm good, thanks!");
  sleep_for(seconds(1));
}

CAF_MAIN()
\end{cpp}

运行这段代码会产生不同的输出。最理想的结果正是我们所预期的:

\begin{shell}
Bob received: Hello Bob!
Alice received: Hello Alice!
Alice received: How are you?
Bob received: I'm good, thanks!
\end{shell}

但反复运行这段代码会产生不同的结果,如下所示:

\begin{shell}
Bob received: Hello Bob!
Bob received: I'm good, thanks!
Alice received: Hello Alice!
Alice received: How are you?
\end{shell}

还可能得到更糟糕的输出结果:

\begin{shell}
Alice received: Hello Alice!
BobAlice received: How are you? received: Hello Bob!

Bob received: I'm good, thanks!
\end{shell}

这些结果清楚地表明,actor 是真正并行运行的。同时,它们也揭示了一个现实:尽管框架在底层做了大量封装,并行编程的复杂性终究无法被完全隐藏。我们仍然需要理解消息传递、状态同步以及潜在的并发问题,只是这些问题现在以更高层次的形式呈现出来。

不过,Actor 模型为我们提供了一种强大的抽象方式 --- 它让我们将并行编程看作是一组独立对象对请求做出响应的过程。这种模型不仅更贴近现实世界的运作方式,还允许我们根据系统需求灵活选择 actor 的类型和通信机制。前面的例子展示了一个基于事件、接收异步消息的 actor,但许多框架(如 CAF)同样支持阻塞式消息传递,以及具有不同生命周期的 actor 类型。

Actor 模型的一大优势在于其天然的可扩展性。我们可以轻松地将 actor 分布在不同的计算节点上,从而构建出分布式系统。这意味着,从第一行代码开始,我们的程序就具备了横向扩展的能力。当然,这也意味着我们立刻面对分布式系统的典型挑战,例如网络延迟、节点故障、数据一致性等问题。

至此,我们已经了解了现代 C++ 标准库所提供的并发支持,以及经过实践验证的 Actor 模型所能实现的功能。然而,在面对日益复杂的软件需求时,我们仍会遇到一些当前语言和标准尚未原生支持的场景。那么,到底还有哪些能力是我们目前无法轻易实现的?这正是下一节要探讨的重点。