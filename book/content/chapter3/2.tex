如果你像我一样,经常在不同的组织、团队和专业技术会议之间穿梭往来,你会很快注意到两件事:C++ 程序员的兴趣与其他开发者截然不同,而且 C++ 社区更准确地说是由一个个小型、专业化的 C++ 开发者群体组成。这与其它语言社区大不相同;如果你谈论 Java,你很可能最终是在讨论 Spring 框架和 REST API,或者 Android 工具包;C\# 主要围绕微软的类库高度标准化;而 JavaScript 则几乎总是离不开 React。但如果你把来自不同组织的一百个 C++ 程序员聚集在一个房间里,你很快就会发现他们之间的差异。嵌入式 C++ 开发者关注的是如何严格控制所有资源,因为对于销量以百万计的设备来说,哪怕增加 1MB 的内存都会迅速推高成本。游戏开发者则处于另一个极端,他们关心的是如何从新一代 GPU 和 CPU 中榨取更多的帧率。高频交易领域的程序员对避免 CPU 缓存未命中了如指掌,并且知道如何从自动化交易算法中“节省”几皮秒(picosecond)的时间,因为最微小的时间单位可能意味着数百万欧元的差距。工程软件开发者则相对宽松一些,但仍会担心复杂渲染模型中更改的正确性。还有那些开发铁路、汽车或工厂自动化系统的程序员,他们的主要关注点是系统的弹性和鲁棒性。

虽然这个图景远非完整,但它已经足以向我们展示 C++ 程序员之间巨大的多样性,这是其他任何语言的开发者群体都无法比拟的。我们甚至可以说,从某种角度来看,C++ 是目前仍在使用的最后一种事实上的通用编程语言,因为其他主流语言在实际应用中大多仅用于特定类型的程序:Java 用于企业后端服务和 Android 开发,C\# 用于 Web 和 Windows 应用及服务,JavaScript 用于富客户端网页和无服务器后端,Python 用于脚本、数据科学和 DevOps。而 C++ 却广泛应用于嵌入式软件、工厂系统、金融交易、仿真、工程工具、操作系统等各个领域。

那句老话“形式追随功能”原本是用来描述人类所建造的一切事物的设计理念,包括编程语言,它同样也适用于 C++。项目和程序员类型的巨大多样性被融入到了语言之中,再加上斯特劳斯特鲁普(Stroustrup)希望它尽可能强大的愿望。

C++ 并不是一门单一的语言;每一位程序员都在使用 C++ 的一个子集,而这往往与其在同一组织工作的同事所使用的部分大相径庭。

没错,C++ 起源于面向对象编程兴起的时代,它最初只是“带有对象的 C”。但与此同时,C++ 又与 C 保持向后兼容,这意味着你仍然可以在 C++ 中编写结构化编程风格的代码。后来模板变得必不可少,再后来 lambda 表达式也变得实用。虽然 C++ 一直以来都是多种编程范式的集合体,但在今天,这种特性更加明显。为了证明这一点,让我们来看看几种你可以在 C++ 中使用的编程范式,首先从函数式编程开始。