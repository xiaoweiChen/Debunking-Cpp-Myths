

\begin{center}
\thispagestyle{empty}
%\includegraphics[width=\textwidth,height=\textheight,keepaspectratio]{cover.png}
\begin{tikzpicture}[remember picture, overlay, inner sep=0pt]
\node at (current page.center)
{\includegraphics[width=\paperwidth, keepaspectratio=false]{cover.png}};
\end{tikzpicture}
\newpage
\thispagestyle{empty}
\huge
\textbf{走出C++谜云}
\\[9pt]
\normalsize
开启一段富有洞见的旅程,揭开 C++ 迷思与误解背后的真相
\\[9pt]
\normalsize
作者: Alexandru Bolboacă, Ferenc-Lajos Deák
\\[8pt]
\normalsize
译者:\href{https://github.com/xiaoweiChen/Debunking-Cpp-Myths}{陈晓伟}
\\[8pt]
\end{center}

\newpage

\begin{comment}
\end{comment}

\pagestyle{empty}
\tableofcontents
\newpage

\setsecnumdepth{section}

\myChapter{致谢}{}{content/dedicated.tex}
\newpage

\myChapter{关于作者}{}{content/about-the-author.tex}
\newpage

\myChapter{关于评审}{}{content/about-the-reviewer.tex}
\newpage

\myChapter{前言}{}{content/preface.tex}
\newpage

\myChapter{第1章}{C++很难学}{content/chapter1/0.tex}
\mySubsection{1.1.}{环境要求}{content/chapter1/1.tex}
\mySubsection{1.2.}{为什么 C++ 被认为很难学习?}{content/chapter1/2.tex}
\mySubsection{1.3.}{C++ 的难点以及如何掌握它们}{content/chapter1/3.tex}
\mySubsection{1.4.}{Stroustrup学习 C++ 的方法}{content/chapter1/4.tex}
\mySubsection{1.5.}{Kate Gregory --- 不要教授C语言}{content/chapter1/5.tex}
\mySubsection{1.6.}{测试驱动学习法}{content/chapter1/6.tex}
\mySubsection{1.7.}{能力越大,责任越大}{content/chapter1/7.tex}
\mySubsection{1.8.}{总结}{content/chapter1/8.tex}
\newpage

\myChapter{第2章}{每个 C++ 程序都应该符合标准}{content/chapter2/0.tex}
\mySubsection{2.1.}{环境要求}{content/chapter2/1.tex}
\mySubsection{2.2.}{在遥远的加纳某地}{content/chapter2/2.tex}
\mySubsection{2.3.}{微软那小巧、柔软的 C++}{content/chapter2/3.tex}
\mySubsection{2.4.}{自由编译器的领域}{content/chapter2/4.tex}
\mySubsection{2.5.}{当头文件甚至都不是 C++ 的时候}{content/chapter2/5.tex}
\mySubsection{2.6.}{被锁在盒子里的 C++ 奇闻}{content/chapter2/6.tex}
\mySubsection{2.7.}{未来 C++ 的峥嵘岁月}{content/chapter2/7.tex}
\mySubsection{2.8.}{总结}{content/chapter2/8.tex}
\newpage

\myChapter{第3章}{有一个C++,它是面向对象的}{content/chapter3/0.tex}
\mySubsection{3.1.}{环境要求}{content/chapter3/1.tex}
\mySubsection{3.2.}{C++的多面性}{content/chapter3/2.tex}
\mySubsection{3.3.}{C++中的函数式编程}{content/chapter3/3.tex}
\mySubsection{3.4.}{元编程}{content/chapter3/4.tex}
\mySubsection{3.5.}{强类型的极限}{content/chapter3/5.tex}
\mySubsection{3.6.}{忽略类型呢?}{content/chapter3/6.tex}
\mySubsection{3.7.}{总结}{content/chapter3/7.tex}
\newpage

\myChapter{第4章}{Main()函数是应用程序的入口点}{content/chapter4/0.tex}
\mySubsection{4.1.}{main()函数}{content/chapter4/1.tex}
\mySubsection{4.2.}{企鹅农场}{content/chapter4/2.tex}
\mySubsection{4.3.}{让我们打开窗户(除非在国际空间站)}{content/chapter4/3.tex}
\mySubsection{4.4.}{总结}{content/chapter4/4.tex}
\newpage

\myChapter{第5章}{C++类中,顺序必须存在}{content/chapter5/0.tex}
\mySubsection{5.1.}{大小很重要}{content/chapter5/1.tex}
\mySubsection{5.2.}{必须尊重的顺序}{content/chapter5/2.tex}
\mySubsection{5.3.}{对顺序的深刻思考}{content/chapter5/3.tex}
\mySubsection{5.4.}{C++的黑暗法则}{content/chapter5/4.tex}
\mySubsection{5.5.}{当顺序不重要的时候}{content/chapter5/5.tex}
\mySubsection{5.6.}{总结}{content/chapter5/6.tex}
\newpage

\myChapter{第6章}{C++不是内存安全的}{content/chapter6/0.tex}
\mySubsection{6.1.}{环境要求}{content/chapter6/1.tex}
\mySubsection{6.2.}{内存安全很重要}{content/chapter6/2.tex}
\mySubsection{6.3.}{旧C++的内存安全问题}{content/chapter6/3.tex}
\mySubsection{6.4.}{现代C++拯救了我们}{content/chapter6/4.tex}
\mySubsection{6.5.}{现代C++的局限性}{content/chapter6/5.tex}
\mySubsection{6.6.}{还有更多的事情要做}{content/chapter6/6.tex}
\mySubsection{6.7.}{总结}{content/chapter6/7.tex}
\newpage

\myChapter{第7章}{C++中实现并行性和并行性并非易事}{content/chapter7/0.tex}
\mySubsection{7.1.}{环境要求}{content/chapter7/1.tex}
\mySubsection{7.2.}{定义并行性和并发性}{content/chapter7/2.tex}
\mySubsection{7.3.}{并行性和并发性的常见问题}{content/chapter7/3.tex}
\mySubsection{7.4.}{函数式编程拯救你!}{content/chapter7/4.tex}
\mySubsection{7.5.}{行动者模型}{content/chapter7/5.tex}
\mySubsection{7.6.}{我们还不能做的事}{content/chapter7/6.tex}
\mySubsection{7.7.}{总结}{content/chapter7/7.tex}
\newpage

\myChapter{第8章}{最快的C++代码是内联汇编}{content/chapter8/0.tex}
\mySubsection{8.1.}{点亮一个像素}{content/chapter8/1.tex}
\mySubsection{8.2.}{所有数字的总和}{content/chapter8/2.tex}
\mySubsection{8.3.}{一条指令统治一切}{content/chapter8/3.tex}
\mySubsection{8.4.}{总结}{content/chapter8/4.tex}
\newpage

\myChapter{第9章}{C++是美丽的}{content/chapter9/0.tex}
\mySubsection{9.1.}{追求美}{content/chapter9/1.tex}
\mySubsection{9.2.}{零的定义}{content/chapter9/2.tex}
\mySubsection{9.3.}{关于圆括号的圆括号}{content/chapter9/3.tex}
\mySubsection{9.4.}{C++uties}{content/chapter9/4.tex}
\mySubsection{9.5.}{总结}{content/chapter9/5.tex}
\newpage

\myChapter{第10章}{C++中没有用于现代编程的库}{content/chapter10/0.tex}
\mySubsection{10.1.}{我们怎么知道呢?}{content/chapter10/1.tex}
\mySubsection{10.2.}{现代开发者的经验}{content/chapter10/2.tex}
\mySubsection{10.3.}{共同的需要}{content/chapter10/3.tex}
\mySubsection{10.4.}{兼容性}{content/chapter10/4.tex}
\mySubsection{10.5.}{供应链安全}{content/chapter10/5.tex}
\mySubsection{10.6.}{总结}{content/chapter10/6.tex}
\newpage

\myChapter{第11章}{C++是向后兼容...即使是C}{content/chapter11/0.tex}
\mySubsection{11.1.}{C真的与C++向前兼容吗?}{content/chapter11/1.tex}
\mySubsection{11.2.}{空格很重要——直到它不再重要}{content/chapter11/2.tex}
\mySubsection{11.3.}{auto的惊喜}{content/chapter11/3.tex}
\mySubsection{11.4.}{总结}{content/chapter11/4.tex}
\newpage

\myChapter{第12章}{Rust将取代C++}{content/chapter12/0.tex}
\mySubsection{12.1.}{环境要求}{content/chapter12/1.tex}
\mySubsection{12.2.}{为什么要竞争?}{content/chapter12/2.tex}
\mySubsection{12.3.}{Rust的核心特性}{content/chapter12/3.tex}
\mySubsection{12.4.}{Rust的优势}{content/chapter12/4.tex}
\mySubsection{12.5.}{C++在哪些方面更好}{content/chapter12/5.tex}
\mySubsection{12.6.}{为什么C++仍然需要}{content/chapter12/6.tex}
\mySubsection{12.7.}{总结}{content/chapter12/7.tex}
\newpage

\begin{comment}
\end{comment}
